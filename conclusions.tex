\section{Conclusions}
This paper has presented a novel video-based side-channel attack for
Android pattern lock. The attack is based on a video filmed a distance of 2 meters away from the target device using a mobile phone camera. The attack is
achieved by employing a computer vision algorithm to track the
fingertip movement from the video, and then using the geometry information
of the fingertip movement trajectory to identify the most likely patterns to be
tested on the target device.  Our approach was evaluated using 120 unique patterns collected
from independent users and some of the most complex patterns. The experimental results show that our approach
is able to successfully crack over 95\% of the patterns in five attempts.
We show that, in contrast to many people's belief, complex pattern actually provides
weaker protection over simple patterns under our attack. Our study
suggests that Android pattern lock is vulnerable to video-based
side-channel attacks.

    %\FIXED{In this paper, we present a novel computer vision based attack that cracks the Android locking patterns based on a video that is unnoticeably filed from a certain angle. Our attacking method uses a video tracking algorithm to track the fingertip motion trajectory, and then outputs a small number of candidate pattern locks by analyzing the line segments and their angle of the fingertip motion.}
%
%    \FIXED{To evaluate this attacking approach, we collect 600 patterns from 215 independent users by asking them to fill an anonymized questionnaires and select 120 unique patterns after removing identical ones for performing various experiments. The experimental results shows that the attack system can unlock the target phone with a success rate of above 95\% with no more than five attempts -- a default threshold when the phone will automatically locked by the Android operating system. Furthermore, this work shows that a complex pattern lock is easier to be cracked than a simple one, a great contradiction in most people's mind that a complex pattern lock is more secure than a simple one. At last, we propose a new insights on how to design and use a pattern lock system in a secure way.}
%
%    %In this paper, we present a novel computer vision based attack that cracks the pattern locks with the unnoticeably filmed video. In this attack, first we film the video from a certain angle. Then we leverages the fingertip movement during the pattern entry process to analyze pattern locks. In order to correct the video filming angles, we use the transformation matrix to map the movement to the victim's perspective. To exclude the similar patterns, we propose a novel representation for pattern locks by graphic knowledge. We implement a prototype of the attack and evaluate it with lots of videos filmed in various situations. The empirical results indicate that we can achieve an average accuracy of 95\% within five trials at two meters away from the touch-screen.
%
%    %However, this attack also exists some limitations which are as following. If user cover his finger during drawing the pattern lock, the accuracy of this attack will be very low and even fails to recognize the pattern lock. Indeed, with the distance increasing, the accuracy will decrease rapidly. Further, when the distance is greater than 3m, the accuracy will be very low. This above limitations is also our future work. Next we will continue on focusing our attention on this limitation and making the attack more robust.
